\documentclass{article}

\usepackage[utf8]{inputenc}
\usepackage{lipsum}
\usepackage{listings}
\usepackage{mdwlist}
\usepackage{fancyhdr}

\pagestyle{fancy}
\fancyhf{}
\fancyhead[R]{Technical University of Denmark}
\fancyhead[L]{02239 -- Data security}
\fancyfoot[R]{\thepage}
\fancyfoot[L]{s070145 -- Jens Henrik Skuldbøl}

\newcommand\Q[1]{
	\leavevmode\par
	\noindent
	\emph{Q -- #1}
	\\
}

\newcommand\A[1]{
	A -- #1
}

\title{Authentication Lab}
\author{Jens Henrik Skuldbøl, s070145}
\date{November 18th, 2013}

\begin{document}

\maketitle
\thispagestyle{empty}

\begin{quote}
	The implemented solutions in this assignment were discussed
	with Brian Lynnerup Pedersen, s042457, during development.
\end{quote}

\newpage

\section{Task 1}

\section{Task 2}

\section{Task 3}
\emph{Please note that this implementation uses Java's `New IO` (\emph{\texttt{java.nio}}) 
and therefor required Java 7 to run.}

The basic architecture of this extension is as follows.
A singleton class maintains a store of <username, password> pairs, read from a file.
Whenever an authentication is attempted, the store is updated.
The first step in an authentication is to check whether the username exists.
If this is the case, the program then encrypts the entered password, concateneted
with the salt of the user, and the resulting string is compared to the stored password value.
If the two match, the user has entered the correct password, and is authenticated.

The structure of the password file is a single line prr user.
This line consists of a username, an (encrypted) password, and a salt, each separated by a semi-colon.
Since the salt and username does not disclose any information on the password, 
these are stored in clear text.

The primary extensions done to this program is in \texttt{sample.module.PasswordStore}.
This class is a singleton, which handles the actual authentication.
Usernames are verified through a call to the method \texttt{validateUsername}.
Once a username is validated, the password is next.
This is validated through a call to \texttt{validatePassword}.
This method then compares the passwords by calling \texttt{testPassword},
which simply returns the result of an equality check of the stored and entered password,
the latter after being encrypted.
The encryption is handled by the \texttt{encrypt} method, which is simply a wrapper around 
the \texttt{md5} method, concatenating password and salt before hashing it.

The use of a message digest, in this case MD5, ensures confidentiality.
Even if an intruder would ever gain access to the password file, the passwords
are not compromised, as the encryption is not reversible.

If an intruder ever did get his hans on a password file, however, he would still be able
to tamper with the file, thereby denying athentic users access to their resources, by
altering the encrypted passwords, or simply deleting users.
A way to safeguard against this would be to restrict access to the file.
Making the file read-only would ensure that no one ever tampers with anything,
but this would have several downsides. Some of these downsides would be that users are unable
to change their passwords. As passwords are not recoverable, users who forgets their
password are then excluded from accessing the system, unless they somehow remembers their password.
Another downside is the lacking support for adding new users.

A way to circumvent these downsides could be to encrypt the password to ensure
authenticity of the author, such as through a asymmetric keypair. 
This could be done by either a trusted, third party, or by
the user him self. While this would not prevent a malicious intruder from destroying the file,
it would allow for easy detection were this the case.
Implementation of this has been considered to be beyond the scope of this excercise.

As the confidentiality of the passwords are ensured, as stated above,
availability is easily obtained by simply making the file readable by everyone.
This would, however, expose a list of usernames to potentially malicious users,
so administrators would have to strike a balance of making the file available
to relevant users, while unavailable to the outside world.
On top of this, it is also bad practise to inform a user of whether a username is correct
on a failed authentication, but as this was the basis of the solution, it has not been removed.
It also serves a debugging purpose in this case.

\section{Running the program}
Included in the handin are two shell files, \texttt{run.sh} and \texttt{build.sh}.
Their function should be fairly obvious from their names.
Though the binaries should be included, \texttt{build.sh} has been included in case
the program should need to be rebuilt.

\texttt{run.sh} runs the program, passing the commands listed in the exercise as
arguments.

Please note that these files are intend for use on a machine running the Debian OS.
No garanties are made, that they will work on other OSs.

\newpage
\appendix

\section{Code}

\subsection{Task 3}

\subsubsection{SampleLoginModule}
\label{code:sampleloginmodule}
This section contains the changed code from the \texttt{SampleLoginModule} class.
\lstinputlisting
[
	basicstyle=\footnotesize,
	numbers=left,
	numberstyle=\tiny,
	breaklines=true,
	language=Java,
	firstline=175,
	lastline=192
]
{../src/sample/module/SampleLoginModule.java}

\subsubsection{PasswordStore}
\label{code:passwordstore}
This is the complete source code for the \texttt{PasswordStore} class.
\lstinputlisting
[
	basicstyle=\footnotesize,
	numbers=left,
	numberstyle=\tiny,
	breaklines=true,
	language=Java
]
{../src/sample/module/PasswordStore.java}
\end{document}
